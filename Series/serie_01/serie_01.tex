\documentclass[12pt,letterpaper]{article}
\usepackage{fullpage}
\usepackage[top=2cm, bottom=4.5cm, left=2.5cm, right=2.5cm]{geometry}
\usepackage{amsmath,amsthm,amsfonts,amssymb,amscd}
\usepackage{lastpage}
\usepackage{enumerate}
\usepackage{enumitem}
\usepackage{fancyhdr}
\usepackage{mathrsfs}
\usepackage{xcolor}
\usepackage{graphicx}
\usepackage{listings}
\usepackage{minted}
\usepackage{hyperref}


\hypersetup{%
  colorlinks=true,
  linkcolor=blue,
  linkbordercolor={0 0 1}
}
 
\renewcommand\lstlistingname{Algorithm}
\renewcommand\lstlistlistingname{Algorithms}
\def\lstlistingautorefname{Alg.}

\lstdefinestyle{C}{
    language        = C,
    frame           = lines, 
    basicstyle      = \footnotesize,
    keywordstyle    = \color{blue},
    stringstyle     = \color{green},
    commentstyle    = \color{red}\ttfamily
}

\lstdefinestyle{bash}{
    language=Bash,
    backgroundcolor=\color{darkgray},
    keywordstyle=\small\color{white},
    commentstyle=\color{Grey},
    stringstyle=\color{Red},
    showstringspaces=false,
    basicstyle=\small\color{white},
    numbers=none,
    captionpos=b,
    tabsize=4,
    breaklines=true
}

\lstdefinestyle{output}{
    backgroundcolor=\color{lightgray},
    basicstyle=\ttfamily,
}

\setlength{\parindent}{0.0in}
\setlength{\parskip}{0.05in}

% Edit these as appropriate
\newcommand\course{IN.2020}
\newcommand\hwnumber{2}                  % <-- homework number
\newcommand\name{Gremaud Lucien}           % <-- NetID of person #1

\pagestyle{fancyplain}
\headheight 35pt
\lhead{\name}
\chead{\textbf{\Large Serie \hwnumber}}
\rhead{\course \\ \today}
\lfoot{}
\cfoot{}
\rfoot{\small\thepage}
\headsep 1.5em

\begin{document}

\section{C}

    \subsection{Word Count - Some Shell Commands} 
    
    \begin{lstlisting}[style = bash]
        % ./wcount
    \end{lstlisting}
    Executes the file "wcount".

    \begin{lstlisting}[style=bash]
        % ./wcount < wcount.c 
    \end{lstlisting}
    Executes the file "wcount" with "wcount.c" as a standard input.
    
    \begin{lstlisting}[style=bash]
        % ./wcount < wcount > test
    \end{lstlisting}
    Executes the file "wcount" with "wcount" as a standard input and "test" as a standard output.
    
    \begin{lstlisting}[style=bash]
        % cat wcount.c | ./wcount 
    \end{lstlisting}
    We use "cat" to display the content of "wcount.c" and insert it as a standard input after the execution of "wcount" with the help of a pipe.
    
    \begin{lstlisting}[style=bash]
        % grep { wcount.c  
    \end{lstlisting}
    Looks for the character "\{" in the file "wcount.c" and display every line containing this character.
    
    \begin{lstlisting}[style=bash]
        % grep { wcount.c | ./wcount
    \end{lstlisting}
    Same as above but we use the output of the first expression as the input of the second expression.
    
    \subsection{Type Conversion, Casting and ASCII}
    
    \begin{lstlisting}[style=output]
    A 65
    A 65
    3.140000 3
    \end{lstlisting}
    On the first line, the character 'A' is first displayed as a character and then as an integer.
    On the second line, the integer 65 is first displayed as a character and then as an integer.
    On the last line, the float "pi" is first displayed as a float and then as an integer, leaving its decimal part.
    
    \subsection{Constant, Variable, Escape Character '\textbackslash', and Octal resp. Hexadecimal Digits}
    \begin{lstlisting}[style=output]
    @ 64 100 40
    @ 64 100 40
    @ 64 100 40
    \end{lstlisting}
    All of the variables ('@', at and AT), are essentially of the same value simply written in different format. So it is expected for them to display the same output if converted in a equivalent format.
    
    \subsection{enum Type}
    \subsection{Logical Expression}
    \subsection{Conditional Expression}

\section*{PART 2. System}

\end{document}